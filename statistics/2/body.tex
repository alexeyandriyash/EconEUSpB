\documentclass[a4paper, 12pt]{extarticle}
\usepackage[T2A]{fontenc}
\usepackage[english,russian]{babel}
\usepackage[utf8]{inputenc}
\usepackage[left=3cm,right=1cm,top=2.5cm,bottom=2 cm,bindingoffset=0cm]{geometry}
\usepackage{subfiles}
\usepackage{amsmath}
\usepackage{amsfonts}
\usepackage{float}
\usepackage{setspace}
\usepackage{listings}
\usepackage{caption}
\usepackage[final]{pdfpages}
\usepackage{url}
\usepackage{graphicx}
\usepackage{indentfirst}
\usepackage{xcolor}
\usepackage{hyperref}
\linespread{1.1}
\pagestyle{myheadings}

\numberwithin{figure}{section}
\graphicspath{{../images/}{images/}}

\definecolor{linkcolor}{HTML}{102C54}
\definecolor{urlcolor}{HTML}{102C54} 

\hypersetup{pdfstartview=FitH, linkcolor=linkcolor,urlcolor=urlcolor, colorlinks=true, citecolor=black}
\begin{document}

\subsection*{Домашняя работа \#2}

%\begin{flushright}
\subsubsection*{Андрияш Алексей Андреевич}
%\end{flushright}

\subsection*{25}
Число сбоев в компьютерной сети, возникающих в течение дня, подчиняется
пуассоновскому закону с параметром $\lambda = 2$ . Найдите вероятности того, что за день\\
а) не будет ни одного сбоя;\\
б) будет ровно 1 сбой;\\
в) будет не менее двух сбоев;\\
г) не более трёх сбоев;\\

Решение:\\

$P(\xi=k)=\frac{\lambda^k}{k!}e^{-\lambda}$\\
\\
a) $\lambda=2$, $k=0$\\
\\
$P(\xi=0)=\frac{2^0}{0!}e^{-2}$ = $\frac{1}{1}e^{-2}$ = $e^{-2}$\\
\\
б) $\lambda=2$, $k=1$\\
\\
$P(\xi=1)=\frac{2^1}{1!}e^{-2}$ = $\frac{2}{1}e^{-2}$ = $2e^{-2}$\\
\\
в) будет не менее двух сбоев;\\
\\
$1-\sum_{k=0}^{2}P(k)$
\\
$P(0)= e^{-2}$\\
$P(1)= 2e^{-2}$\\
$P(2)=\frac{2^2}{2!}e^{-2}$ = $\frac{4}{2}e^{-2}$ = $2e^{-2}$\\
$1-\sum_{k=0}^{2}P(k) = 1-(e^{-2} + 2e^{-2} + 2e^{-2}) = 1 - 5e^{-2}$\\
\\
$1-\sum_{k=0}^{2}P(k)$ = $1 - 5e^{-2}$\\
\\
г)не более трёх сбоев;\\
\\
$\sum_{k=0}^{3}P(k)$
\\
$P(0)= e^{-2}$\\
$P(1)= 2e^{-2}$\\
$P(2)= 2e^{-2}$\\
$P(3)=\frac{2^3}{3!}e^{-2}$ = $\frac{8}{6}e^{-2}$ = $\frac{4}{3}e^{-2}$\\
\\
$\sum_{k=0}^{3}P(k) = \frac{19}{3}e^{-2}$
\\


\subsection*{32}
Приведите аналог формул (*) и (**) для $F_\eta$ и $f_\eta$ для случая строго убывающей непрерывной функции $g$.\\

Решение:\\
\\
(*) $F_\eta(x) = P_\eta(g(\eta)>x)=P_\eta(\eta>g^{-1}(x)) = 1 - F_\eta(g^{-1}(x))$, для $g'(x) < 0$\\
\\
(**) $f_\eta(x) = - f_\eta(g^{-1}(x))(g^{-1}(x))'$, для $g'(x) < 0$\\
\\


\subsection*{33}
Случайная величина $\xi$ распределена нормально с параметрами $\mu$ и $\sigma^2$. Найдите плотность распределения случайной величины $\eta = e^\xi$.\\

Решение:\\

$F_\xi(x) = \frac{1}{\sqrt{2\pi}\sigma}e^{-\frac{(x-\mu)^2}{2\sigma^2}}$\\
\\
(**) $f_\eta(x) = f_\eta(g^{-1}(x))|(g^{-1}(x))'|$\\

$f_\eta = F_\eta(\frac{1}{\sqrt{2\pi}\sigma}e^{-\frac{(x-\mu)^2}{2\sigma^2}}))' = (e^{\frac{1}{\sqrt{2\pi}\sigma}e^{-\frac{(x-\mu)^2}{2\sigma^2}}})'$\\
\\
\\
$\frac{x-\mu}{\sqrt{2\pi}\sigma^3}$
$e^{\frac{1}{\sqrt{2\pi}\sigma}e^{-\frac{(x-\mu)^2}{2\sigma^2}}}$
$e^{-\frac{-(x-\mu)^2}{2\sigma^2}}$
\\


\subsection*{35}
Найдите функцию квантилей экспоненциального распределения с параметром $\lambda$.\\
Найдите медиану и 5\%-ую точку.\\

Решение:\\
\\
$F(x)=\lambda*e^{-\lambda*x}$\\
$Q(\alpha) = F^-1(x) = \frac{1}{\lambda}e^{\lambda*\alpha}$ , $\lambda>0$\\
$Q(0.5) = \frac{1}{\lambda}e^\frac{\lambda}{2}$, $\lambda>0$\\
\\
$Q(0.95) = \frac{1}{\lambda}e^{\lambda*0.95}$, $\lambda>0$\\
\end{document} 

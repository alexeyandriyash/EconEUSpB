\documentclass[a4paper, 12pt]{extarticle}
\usepackage[T2A]{fontenc}
\usepackage[english,russian]{babel}
\usepackage[utf8]{inputenc}
\usepackage[left=3cm,right=1cm,top=2.5cm,bottom=2 cm,bindingoffset=0cm]{geometry}
\usepackage{subfiles}
\usepackage{amsmath}
\usepackage{amsfonts}
\usepackage{float}
\usepackage{setspace}
\usepackage{listings}
\usepackage{caption}
\usepackage[final]{pdfpages}
\usepackage{url}
\usepackage{graphicx}
\usepackage{indentfirst}
\usepackage{xcolor}
\usepackage{hyperref}
\linespread{1.1}
\pagestyle{myheadings}

\numberwithin{figure}{section}
\graphicspath{{../images/}{images/}}

\definecolor{linkcolor}{HTML}{102C54}
\definecolor{urlcolor}{HTML}{102C54} 

\hypersetup{pdfstartview=FitH, linkcolor=linkcolor,urlcolor=urlcolor, colorlinks=true, citecolor=black}
\begin{document}

\subsection*{Домашняя работа \#2}

%\begin{flushright}
\subsubsection*{Андрияш Алексей Андреевич}
%\end{flushright}

\subsection*{25}
Число сбоев в компьютерной сети, возникающих в течение дня, подчиняется
пуассоновскому закону с параметром $\lambda = 2$ . Найдите вероятности того, что за день\\
а) не будет ни одного сбоя;\\
б) будет ровно 1 сбой;\\
в) будет не менее двух сбоев;\\
г) не более трёх сбоев;\\

\subsection*{32}
Приведите аналог формул (*) и (**) для $F_\eta$ и $f_\eta$ для случая строго убывающей непрерывной функции $g$.\\

\subsection*{33}
Случайная величина $\xi$ распределена нормально с параметрами $\mu$ и $\sigma^2$. Найдите плотность распределения случайной величины $\eta = e^\xi$.\\

\subsection*{35}
Найдите функцию квантилей экспоненциального расперделения с параметром $\lambda$.\\
Найдите медиану и 5\%-ую точку.\\


\end{document} 

\documentclass[a4paper, 12pt]{extarticle}
\usepackage[T2A]{fontenc}
\usepackage[english,russian]{babel}
\usepackage[utf8]{inputenc}
\usepackage[left=3cm,right=1cm,top=2.5cm,bottom=2 cm,bindingoffset=0cm]{geometry}
\usepackage{subfiles}
\usepackage{amsmath}
\usepackage{amsfonts}
\usepackage{float}
\usepackage{setspace}
%\usepackage{listings}
\usepackage{caption}
\usepackage[final]{pdfpages}
\usepackage{url}
\usepackage{graphicx}
%\usepackage{indentfirst}
\usepackage{xcolor}
\usepackage{hyperref}
\linespread{1.5}
\pagenumbering{gobble}

\numberwithin{figure}{section}
\graphicspath{{../images/}{images/}}

\definecolor{linkcolor}{HTML}{102C54}
\definecolor{urlcolor}{HTML}{102C54} 

\hypersetup{pdfstartview=FitH, linkcolor=linkcolor,urlcolor=urlcolor, colorlinks=true, citecolor=black}
\begin{document}


\subsubsection*{Андрияш Алексей Андреевич}
\subsection*{Краткое резюме доклада к научной статье}
\subsubsection*{Голощапова И., Андреев М. (2017) Оценка инфляционных ожиданий российского населения методами машинного обучения, Вопросы экономики, №6, С. 71-93}

Представленная выше статья является результатом проведенной НИОКР. Голощапова И., Андреев М. применили новейшие технологические возможности для сбора экономически релевантных данных, тем самым находясь на стыке нескольких дисциплин. Авторы используют цифровую среду как контейнер объекта экономического исследования, а методы машинного обучения, анализа как инструмент сбора данных в этой среде.

Исследователи ставят своей целью подсчёт индикаторов Интенсивности и Неопределенности инфляционных ожиданий, на основе анализа комментариев пользователей, к значимым тематическим статьям от ведущих российских изданий. В качестве референсных методов применяются классические опросные методы, не приспособленные к цифровой среде.

Перед Голощаповой И., Андреевым М. встает ряд сложных вопросов и задач, а именно: как отбирать источники данных?; как определить, носит ли произвольный единичный комментарий тематичести релевантную информацию?; как извлечь эту информацию? как конвертировать её в целевые индексы? как агрегировать индексы, полученные из десятков тысяч комментариев? Кроме того, красной нитью, через всю статью проходит вопрос сложности технической реализации поставленной задачи. 

Само собой, как и в любой НИОКР, авторами целенаправленно сделан ряд допущений и упрощений, снижающих академическую строгость работы, а вместе с тем и сложность. Однако, позволивших быстро и гибко дать оценку новому методу. Описать метод, границы его применимости, преимущества и возможности к совершенствованию. 

Для решения поставленной экономической задачи, исследователи использовали последние достижения из области психологии, математики и программирования. Провели емкую аналитическую и техническую работу. Статья, без сомнения, является интересным объектом для обсуждения, а кроме того открывает ряд новых вопросов.


\end{document} 
